\section{Data processing and basic results}
\label{section:results}

%\begin{figure*}[ht!]
%\centering
%\includegraphics[width=0.8\textwidth]{graphics/sequence_progress.pdf}
%\caption{TCP sequence progress in both directions}
%\label{fig:tcp}
%\end{figure*}

To process the PCAP file we have converted the data into
JSON format using Wireshark application, and later parsed 
the file using custom Python script. We then extracted 
the sequence numbers, length of the TCP payload, 
source and destination ports. 

The result of the sequence progress is shown in Figure~\ref{fig:tcp}.

\begin{figure}[h]
\centering
\includegraphics[width=0.4\textwidth]{graphics/sequence_progress.pdf}
\caption{TCP sequence progress in both directions}
\label{fig:tcp}
\end{figure}

Clearly, drop of the spikes indicate TCP connection restarts. But one thing
is really interesting, the last TCP connection restarts successfully, but the sequences
are progressing only for one direction: from client to server. The data does not go 
through the CGN from server to client and the TCP connection hangs. 

