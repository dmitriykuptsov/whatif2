\chapter{Introduction}

\section{Questions}

In this section we enumerate research questions that we would like to 
answer in this report. We foresee at least three important questions
in this report. Namely those are: {\bf (i) What to measure in the enterprise networks? }
By answering this question we attempt to shed the light
on most important characteristics in the network traffic analysis. Be it
latency, throughput, goodput, error and loss rate, availability.
{\bf (ii) How to measure?} Here we would like to answer how to perform the network measurements.
For example, how to select vantage point, how to minimize the 
dataset, but still be able to grasp the most important characteristics of
the network. {\bf (iii) When to measure?} This question is also important for a number of reasons.
Selecting correct measurement period and correct duration of the measurements intervals
will have the important impact on the quality of the research outcome.

By answering these questions properly one can illuminate the performance of the 
networking infrastructure. Note, in this work we are considering only small networks, comprising 
10-100 devices. However, we believe that these questions are also applicable to larger
networks and more complicated topologies.

\chapter{Background}

This section consists of several important parts: First, we 
discuss main network characteristics of the network; Second,
we describe various types of network topologies and device 
orchestrations; Third, we present various tools which are useful
in enterprise network measurements; Finally, we discuss most wide spread 
network protocols to watch out for in the enterprise 
network traffic.

\section{Basic network characteristics}

We believe that there are several key network characteristics: delay, jitter, throughput, goodput, error and loss rates.
All these metrics can be used to measure the performance of the networked systems. In the paragraphs that follow
we will describe these metrics and try to explain why they are so important.

{\it Delay } is the time it takes for the packet (on network layer) to reach the other communication side. Delay can be 
one way or two way, also called {\it round trip time}. The former one is hard to measure since the clocks on both sides
need to be synchronized. Technically, of course we can use sophisticated algorithms and packet trains to measure one way 
delay (for example, the reader can look at the RFC 7679~\cite{}). But most of the time people rely on half of the RTT. This 
metric is less accurate since the packets can travel different paths and, hence, delays can be different. But yet this metric 
is quite common. For example, {\it ping} utility reports RTT as the measure of delay. We should note that delay impacts 
user experience greatly, and therefore, it is good to have links with low delays.

{\it Jitter} is yet another important metric and is wide spread across network engineers. Jitter is the variation of the 
delay, that is jitter shows how much the delay is varying throughout time. This metric is important because it can 
affect how the protocol timers are calculated. For example, if the jitter is high, the calculated timers can be 
inaccurate and, hence, the performance of such protocols can be undermined. Thus, the lower the jitter the better, in 
authors opinion, the performance of the networked devices. One easy way to compute jitter is to build the histogram 
of the network delays and compute the variance.

{\it Throughput} is also important and it captures how much data (including protocol headers and user payload) 
 can be delivered throughout network system in predefined time interval. Typically, the throughput is measured
in Kb/s, Mb/s and Gb/s. Obviously, the larger is the throughput the better network operates. Networks with large 
throughput can service larger number of clients. {\it Goodput} is the same as throughput, but excludes 
the control data from the calculations. In other words, packet header is excluded from the calculations and only 
user's payload is considered.

{\it Error rate} describes how often the packets arrive at the receiver with the corrupted bits. Most of the protocols
use notion of reliability (that is if the data is corrupted it is requested again) and, hence, high error rate can 
reduce the performance of such protocols considerably. It is therefore important for the network engineers to avoid
highly unstable links. Different media and operational environments have different error rates. For example, wireless
links are often have grater error rate than wired and optical links. Also, different applications have different 
tolerance to errors. For example, Voice over IP and Video over IP require error rates to be low. Mail systems, on the 
other side, can tolerate high error rates, because the system works in the background and corrupted packets can be 
requested again.

{\it Loss rate} is the final metrics that we will cover. Loss occurs, for example, when intermediary routers 
drop the packet because of congestion and corruption of the packet. Once again, similar to error rates, sensitive 
applications do not operate well in lossy environments. Hence, typically, network engineers design systems so that
such applications will use links with little loss, while other traffic such as HTTP and SNMP protocols can use 
less expensive, but yet, lossy links. Typically, congested and wireless links with weak signals and obstructed 
with concrete, for example, have higher loss rates than fat wired and optical links.

There are also other characteristics that can be measured, for example, the reader can take a look at the 
congestion and some other. But we are not going to cover those in this report.

\section{Topologies}

There are several key topologies that are used in enterprises: mesh, star and hybrid. {\it Mesh} 
topology is such topology in which every network element is connected to every other network element. The links can be 
wired, wireless and pseudo links (if we are talking about overlays). Mesh can be full and partial.
In full mesh, obviously, ever element is connected every other element in the network. Consider,
for example, personal area network in which all nodes are connected using wireless medium. 
Wired meshes are expensive, though, and are rarely used in modern deployments. Meshes are 
crucial, however, when availability is a must. In mesh, some nodes can fail, yet, the network
will remain alive and packets will be delivered, not to all, but some devices at least. 

{\it Star} topologies are cheaper and less fault tolerant. In star-like topology some node becomes 
the root, while others connected to the root element and all the traffic flows through it. 
Oftentimes, redundant links are added to the topology to bring some level of tolerance to failures.
A typical star topology is shown in Figure~\ref{}. 

Finally, there are also hybrid topologies, such as {\it hub-and-spoke} networks. In this type of
networks spoke nodes are connected to hubs, while hubs form a full mesh between each other. Such
networks are cheaper than full mesh, but more fault tolerant than star topologies. A typical network
is shown in Figure~\ref{}.

\section{Tools}

Network engineers use wide variety of tools for measuring the permanence of the networks and in debugging tasks.
All tools can be categorized based on tasks they are meant to be used for such as measuring and troubleshooting.

{\it Measurement tasks:} to measure performance of the network a common set of tools includes ping utility, tcpping utility, traceroute utility, nc tool,
speedtest and iperf, snmp statistics reports by agents and direct queries. In the following paragraph we will describe these tools.

{\it Ping} utility is the most common tool to measure reachability of a host and measure the round trip times. The tool is based 
on the ICMP protocol which we will describe later. This tool is widely available in Linux, Windows, BSD and Unix operating 
systems.

{\it TCPing} is another common network tool to measure the reachability of the TCP port in the network. But the tool can be also used 
to measure the time to establish a TCP connection. To our best knowledge this tool is widely available for Windows OS without any 
charges.

{\it speedtest and iperf} utilities were primarily designed for measuring the bandwidth between two systems. The later one has
client and server implementations. This means that to measure the bandwidth (both UDP and TCP connections can be used) one needs
to start first the server with the -s flag, and only then run the client with -c flag.

{\it treceroute} is used often to trace the path the packet takes from host A to host B. The tool uses ICMP under the hood. 
Not only it is used to trace the reachability of the intermediate routers but it also reports the RTTs.

{\it netcat, or nc} is the tool that is available in Linux distributions and commonly used to send the commands to the server over UDP sockets.
It can be used to measure the bandwidth. For example, network administrators can send large enough binary package to the server and measure the 
time it takes for the transmission to happen.

And finally {\it SNMP} can be used to collect reports from agents about network performance. Such characteristics as bytes per second 
delivered by the network interface, ping RTT and many more can all be collected from network devices either with direct queries
or with the help of agents.

{\it Trouble shooting}. A set of tools available in this category. nmap, tcpdump, Wireshark, nslookup, dig, iproute, and ifconfig. All these
tools are indispensable in analyzing and troubleshooting the network problems. It is essential for any network engineer to be acquainted 
and actively apply these tools. In the paragraphs that follows we will describe what every tool means, but briefly.

{\it nmap} is a tool that allows network engineer to detect open network ports (both UDP and TCP). Essentially, the tool scans the 
remote system and reports which port is open. Network engineers and hackers actively use this tool to detect weak points in the 
network systems.

{\it tcpdump and Wireshark}. These tools are helpful in collecting and analyzing network protocols. tcpdump is typically used
to collect the raw packets and frames from the network interface of interest, while Wireshark is more advanced it can analyze the
network traffic and export it to XML and JSON file formats. We ourselves use these tools in our network traffic analysis tasks.

{\it nslookup and dig} are used to detect the problems with the DNS servers and queries. These tools are often used by network 
engineers to detect the problems with DNS and get information about remote systems. 

{\it iproute} is a tool that can be used for multiple purposes, but primarily this tools is useful in configuring the routes to remote 
systems. For example, network engineers can use the tool to check whether the routes to remote hosts exist. It can be also used to add
static routes. But these are just few examples. 

And finally, {\it ifconfig} tool can be used to configure the network interfaces. For example, network engineers can use this tool
to set default gateway, IP address on the interface, and set the DNS server name, either manually or with help of DHCP server.  
 
\section{Network protocols to watch out for}

\subsection{Spanning trees}

\subsection{TCP, UDP and ICMP}

\subsection{DNS and DHCP}

\subsection{LDAP and other Windows services}

\subsection{What about timing: NTP}

\subsection{SSH, TLS and other security protocols}

\subsection{Watch out for anti-virus}

\chapter{Results}

\section{Breakdown of the network protocols in small enterprise network}

\section{Watch out! We are plotting the network map}

\section{Performance, performance and once again performance: Stressing the outside world}

\section{Looking for the world: CGNs and TCP}

\chapter{Conclusions}

